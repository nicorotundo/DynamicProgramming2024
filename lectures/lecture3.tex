\documentclass[11pt, aspectratio=169]{beamer}

\usepackage{amsmath, amsfonts, microtype, nicefrac, amssymb, amsthm, centernot}

\usepackage{pgfpages}

\usepackage{helvet}
\usepackage[default]{lato}
\usepackage{array}

\usefonttheme[onlymath]{serif}

\usepackage[utf8]{inputenc}
\usepackage[T1]{fontenc}
\usepackage{textcomp}
\usepackage{bm}

\usepackage{mathpazo}
\usepackage{hyperref}
\usepackage{multimedia}
\usepackage{graphicx}
\usepackage{multirow}
\usepackage{graphicx}
\usepackage{dcolumn}
\usepackage{bbm}
\newcolumntype{d}[0]{D{.}{.}{5}}

\usepackage{graphicx}
\usepackage[space]{grffile}
\usepackage{booktabs}

\usepackage{setspace}

\usepackage{transparent}


%%% FIGURES %%%
\usepackage{caption, subcaption}
\usepackage{booktabs, siunitx}
\usepackage{pgfplots} 
%\usepackage[outdir=./figures]{epstopdf}
\usepackage{float}
\usepackage{graphicx}
\usepackage[absolute, overlay]{textpos}
\usepackage{epstopdf}


%%% TIKZ %%%
\usepackage{tikz}
\usepackage{verbatim}
\usetikzlibrary{arrows.meta}
\usetikzlibrary{positioning}
\usetikzlibrary{bending}
\usetikzlibrary{snakes}
\usetikzlibrary{calc}
\usetikzlibrary{arrows}
\usetikzlibrary{decorations.markings}
\usetikzlibrary{shapes.misc}
\usetikzlibrary{matrix, shapes, arrows, fit, tikzmark}


%%% ALGORITHM %%%
\usepackage{algorithm}
\usepackage[noend]{algpseudocode}
\usepackage{multimedia}


%%% APPENDIX SLIDE NUMBERING %%%
\usepackage{appendixnumberbeamer}


%%% BEAMER BUTTON %%%
%\setbeamertemplate{button}{\tikz
	%\node[
	%	inner xsep = 2pt, 
	%	draw = structure!0, 
	%	fill = myblue, 
	%	rounded corners = 4pt]{\color{white} \tiny\insertbuttontext};
	%}


%%% COLORS %%%
\definecolor{blue}{RGB}{0,38,118}
\definecolor{red}{RGB}{213,94,0}
\definecolor{yellow}{RGB}{240,228,66}
\definecolor{green}{RGB}{0,158,115}

\definecolor{myred}{RGB}{163,32,45}
\definecolor{navyblue}{rgb}{0.05,0.2,0.70}
\definecolor{myblue}{RGB}{0,51,150}
\definecolor{myorange}{RGB}{255,140,0}
\definecolor{myref}{RGB}{160,160,160}
\definecolor{shock}{RGB}{0, 125, 34}%{50, 168, 82}

\definecolor{background}{RGB}{255,253,218}

% Define a new transparent color
\definecolor{trans}{rgb}{1,1,1}
\colorlet{trans}{black!20} % 0 percent opacity

\hypersetup{
  colorlinks=false,
  linkbordercolor = {white},
  linkcolor = {blue}
}

\setbeamercolor{frametitle}{fg=blue}
\setbeamercolor{title}{fg=black}
\setbeamertemplate{footline}[frame number]
\setbeamertemplate{navigation symbols}{} 
\setbeamertemplate{itemize items}{-}
\setbeamercolor{itemize item}{fg=blue}
\setbeamercolor{itemize subitem}{fg=blue}
\setbeamercolor{enumerate item}{fg=blue}
\setbeamercolor{enumerate subitem}{fg=blue}
\setbeamercolor{button}{bg=background, fg=blue,}

%\setbeamercolor{background canvas}{bg=background}


%%% FRAME TITLE %%%
\setbeamerfont{title}{series=\bfseries, parent=structure}
\setbeamerfont{frametitle}{series=\bfseries, parent=structure}


%%% TRANSITION FRAME %%%
\newenvironment{transitionframe}{
	\setbeamercolor{background canvas}{bg=blue}
	\begin{frame}
		\thispagestyle{empty}
		\addtocounter{framenumber}{-1}
		\vspace{42mm}
		\hspace{4mm} }{
		\begin{tikzpicture}
			\tikz \fill [white] (1,6) rectangle (20,10);
		\end{tikzpicture}
	\end{frame}
}


%%% OUTLINE %%%
\AtBeginSection[]
{
	\begin{frame}
       \frametitle{Roadmap of Talk}
       \tableofcontents[currentsection]
   \end{frame}
}
\setbeamercolor{section in toc}{fg=blue}
\setbeamercolor{subsection in toc}{fg=red}
\setbeamersize{text margin left=1em,text margin right=1em} 


%%% ENVIRONMENTS
\newenvironment{witemize}{\itemize\addtolength{\itemsep}{10pt}}{\enditemize}

\makeatother
\setbeamertemplate{itemize items}{\large\raisebox{0mm}{\textbullet}}
\setbeamertemplate{itemize subitem}{\footnotesize\raisebox{0.15ex}{--}}
\setbeamertemplate{itemize subsubitem}{\Tiny\raisebox{0.7ex}{$\blacktriangleright$}}

\setbeamertemplate{enumerate item}[default]
\setbeamertemplate{enumerate subitem}{\textbullet}
\makeatletter

% ITEMIZE SPACING:
% \usepackage{xpatch}
% \xpatchcmd{\itemize}
% {\def\makelabel}
% {\setlength{\itemsep}{0mm}\def\makelabel}
% {}
% {}


%%% PRETTY ENUMERATE %%%
% \usepackage{stackengine,xcolor}
% \newcommand\circnum[2]{\stackinset{c}{}{c}{.1ex}{\small\textcolor{white}{#2}}%
	% 	{\abovebaseline[-.7ex]{\Huge\textcolor{#1}{$\bullet$}}}}
% \newenvironment{myenum}
% {\let\svitem\item
	% 	\renewcommand\item[1][black]{%
		% 		\refstepcounter{enumi}\svitem[\circnum{##1}{\theenumi}]}%
	% 	\begin{enumerate}}{\end{enumerate}}
\usepackage{stackengine,xcolor,graphicx}
\newcommand\circnum[2]{\smash{\stackinset{c}{}{c}{.2ex}{\small\textcolor{white}{#2}}%
		{\abovebaseline[-1.1ex]{\Huge\textcolor{#1}{\scalebox{1.5}{$\bullet$}}}}}}
\newenvironment{myenum}
{\let\svitem\item
	\renewcommand\item[1][black]{%
		\refstepcounter{enumi}\svitem[\circnum{##1}{\theenumi}]}%
	\begin{enumerate}}{\end{enumerate}}

\newcommand{\notimplies}{\;\not\!\!\!\implies}



%%%%%%%%%%%%%%%%%%%%%%%%%%  TITLE   %%%%%%%%%%%%%%%%%%%%%%%%%%%%%%%%
\title[]{\\[8pt]
	{\large \color{blue} Dynamic Programming and Applications \\[5pt] \normalfont{Deterministic Dynamic Programming in Continuous Time} \\[10pt] \normalfont{Lectures 3 -- 4}}}
\author[Schaab]{Andreas Schaab}
\institute{}
\subject{}
\date{}



%%%%%%%%%%%%%%%%%%%%%%%%  BEGIN DOC   %%%%%%%%%%%%%%%%%%%%%%%%%%%%%%%
\begin{document}

%%% TIKZ %%% 
\tikzstyle{every picture}+=[remember picture]
%\everymath{\displaystyle}

\tikzset{   
	every picture/.style={remember picture,baseline},
	every node/.style={anchor=base,align=center,outer sep=1.5pt},
	every path/.style={thick},
}
\newcommand\marktopleft[1]{%
	\tikz[overlay,remember picture] 
	\node (marker-#1-a) at (-.3em,.3em) {};%
}
\newcommand\markbottomright[2]{%
	\tikz[overlay,remember picture] 
	\node (marker-#1-b) at (0em,0em) {};%
}
\tikzstyle{every picture}+=[remember picture] 
\tikzstyle{mybox} =[draw=black, very thick, rectangle, inner sep=10pt, inner ysep=20pt]
\tikzstyle{fancytitle} =[draw=black,fill=red, text=white]


\addtocounter{framenumber}{-1}
\thispagestyle{empty}
\maketitle 
\newpage




%%%%%%%%%%%%%%%%%%%%%%%%%%  SLIDE   %%%%%%%%%%%%%%%%%%%%%%%%%%%%%%%%
\begin{frame}{Outline}
\thispagestyle{empty}
\addtocounter{framenumber}{-1}

Part 1: Differential equations
\begin{enumerate}
	\item The continuous time limit
	\item Ordinary differential equations (ODEs)
	\item Boundary conditions 
	\item Linear first-order ODEs
	\item Examples of ODEs in macro
	\item Application: solving the Solow growth model
	\item Partial differential equations (PDEs)
\end{enumerate}

\end{frame}


%%%%%%%%%%%%%%%%%%%%%%%%%%  SLIDE   %%%%%%%%%%%%%%%%%%%%%%%%%%%%%%%%
\begin{frame}{Outline}
\thispagestyle{empty}
\addtocounter{framenumber}{-1}

Part 2: Optimization with deterministic dynamics 
\begin{enumerate}
	\item Neoclassical growth model in continuous time
	\item Calculus of variations
	\item Optimal control theory
	\item Simple example
	\item Hamilton-Jacobi-Bellman (HJB) equation
	\item First-order condition for consumption 
	\item Envelope condition and Euler equation
	\item Connection between calculus of variations / optimal control and HJBs
	\item Boundary conditions: no-borrowing in the wealth / capital dimension
\end{enumerate}

\end{frame}


%%%%%%%%%%%%%%%%%%%%%%%%%%  SLIDE   %%%%%%%%%%%%%%%%%%%%%%%%%%%%%%%%
\begin{frame}{Outline}
\thispagestyle{empty}
\addtocounter{framenumber}{-1}

Part 3: Applications
\begin{enumerate}
\item Labor: search and matching
\item Urban / trade / dynamic spatial: migration
\item Macro: sticky prices
\item IO: duopoly
\item Public finance: tax competition
\end{enumerate}

\end{frame}


%%%%%%%%%%%%%%%%%%%%%%%%%%  SLIDE   %%%%%%%%%%%%%%%%%%%%%%%%%%%%%%%%
\begin{transitionframe}
	{\color{white} \Huge \textbf{Part 1: Differential Equations} \vspace{2mm}}
\end{transitionframe}


%%%%%%%%%%%%%%%%%%%%%%%%%%  SLIDE   %%%%%%%%%%%%%%%%%%%%%%%%%%%%%%%%
\begin{frame}{1. Continuous time limit}
\begin{witemize}
\item Consider the two key difference equations:
\begin{align*}
	K_{t+1} = I_t + (1-\delta) K_t 
\end{align*}
and
\begin{align*}
	\frac{1}{C_t} = \beta R_t \frac{1}{C_{t+1}}
\end{align*}

\item On the board: (i) generalized discrete time step $\Delta$ and (ii) continuous time limit

\end{witemize}
\end{frame}


%%%%%%%%%%%%%%%%%%%%%%%%%%  SLIDE   %%%%%%%%%%%%%%%%%%%%%%%%%%%%%%%%
\begin{frame}{2. Ordinary differential equations}
\begin{witemize}
\item Consider the ``discrete-time'' equation 
\begin{equation*}
	X_{t+\Delta t} - X_t = G(X_t, t, \Delta t)
\end{equation*}

\item \textit{Continuous-time limit}: consider the limit as $\Delta t \to 0$
\begin{equation*}
	\dot X_t \equiv \frac{dX}{dt} \equiv \lim_{\Delta t \to 0} \frac{X_{t+\Delta t} - X_t}{\Delta t} = \lim_{\Delta t \to 0} \frac{1}{\Delta t} G(X_t, t, \Delta t) \equiv g(X_t, t)
\end{equation*}

\item $\dot X_t = g(X_t)$ is \textit{autonomous} and dropping subscripts: $\dot X = g(X)$

\item This is a \textit{first-order (ordinary) differential equation}, second-order equations are:
\begin{equation*}
	\frac{d^2 X_t}{dt^2} = g \bigg( \frac{dX_t}{dt} , X_t, t \bigg)
\end{equation*}

\item We often consider ODEs in the \textit{time dimension} but ODEs can be defined on any state space (e.g., space dimensions)

\end{witemize}
\end{frame}



%%%%%%%%%%%%%%%%%%%%%%%%%%  SLIDE   %%%%%%%%%%%%%%%%%%%%%%%%%%%%%%%%
\begin{frame}{3. Boundary conditions}
\begin{witemize}
\item Boundary conditions are critical for characterizing differential equations

\item Consider an ODE on the time interval $t \in [0, 1]$. We call $[0, 1]$ the \textit{state space}. $(0, 1)$ is the \textit{interior of the state space} and $\{0, 1\}$ is the \textit{boundary}

\item The way to think about it: differential equations are defined on the interior of the state space but not on the boundary

\item To characterize the function that satisfies the ODE on the interior on the \textit{full} state space, we need a set of boundary conditions to also characterize the behavior on the boundary

\item Heuristically: we need as many boundary conditions as the order of the differential equation
\end{witemize}
\end{frame}


%%%%%%%%%%%%%%%%%%%%%%%%%%  SLIDE   %%%%%%%%%%%%%%%%%%%%%%%%%%%%%%%%
\begin{frame}{}
\begin{witemize}
\item Similar to discrete-time difference equations: forward equations have initial conditions, backward equations have terminal conditions

\item For ODEs, you will often see the terminology:
\begin{itemize}
	\item \textit{Initial value problems} specify a differential equation for $X_t$ with some \textit{initial condition} $X_0$
	
	\item \textit{Terminal value problems} instead specify $X_T$
\end{itemize}

\item More broadly: We need sufficient information to characterize the function of interest along the boundary

\item Types of boundary conditions: Dirichlet ($X_0 = c$), von-Neumann ($\frac{dX_0}{dt} = c$), reflecting boundaries, ...

\item Boundary conditions are very important and can be very subtle (especially for PDEs)
\end{witemize}
\end{frame}


%%%%%%%%%%%%%%%%%%%%%%%%%%  SLIDE   %%%%%%%%%%%%%%%%%%%%%%%%%%%%%%%%
\begin{frame}{4. Linear first-order ODEs}
\begin{witemize}
\item Consider the equation:
\begin{equation}\label{eq:ODE}
	\dot X(t) = a(t) X(t) + b(t)
\end{equation}

\item If $b(t) = 0$, \eqref{eq:ODE} is a \textit{homogeneous} equation, if $a(t) = a$ and $b(t) = b$ we say \eqref{eq:ODE} has \textit{constant coefficients}

\item Start with $\dot X(t) = a X(t)$, divide by $X(t)$ and integrate with respect to $t$
\begin{align*}
	\int \frac{\dot X(t)}{X(t)} dt &= \int a dt \\
	\log X(t) + c_0 &= a t + c_1 \\
	X(t) &= C e^{a t}
\end{align*}
where $C = e^{c_1 - c_0}$

\item Pin down constant $C$ by using the boundary condition (we need $1$)
\end{witemize}
\end{frame}



%%%%%%%%%%%%%%%%%%%%%%%%%%  SLIDE   %%%%%%%%%%%%%%%%%%%%%%%%%%%%%%%%
\begin{frame}{}
\begin{witemize}
\item Consider time-varying coefficient with $\dot X(t) = a(t) X(t)$ with initial condition $X(0) = \bar x$

\item Dividing by $X(t)$, integrating, and exponentiating yields 
\begin{equation*}
	X(t) = C e^{ \int_0^t a(s) ds }
\end{equation*}

\item Constant of integration again pinned down by boundary condition: $C = \bar x$

\item Finally, for $\dot X(t) = a X(t) + b$, we find
\begin{equation*}
	X(t) = - \frac{b}{a} + C e^{at}
\end{equation*}
after using change of variables $Y(t) = X(t) + \frac{b}{a}$

\item Many results for systems of linear differential equations: $\dot{\bm X}(t) = \bm A \bm X(t)$

\end{witemize}
\end{frame}


%%%%%%%%%%%%%%%%%%%%%%%%%%  SLIDE   %%%%%%%%%%%%%%%%%%%%%%%%%%%%%%%%
\begin{frame}{5. Examples of differential equations in macro}

\textbf{Capital accumulation:}
\begin{equation*}
\dot K_t = I_t - \delta K_t
\end{equation*}
\begin{witemize}
\item We can always map back and forth between DT and CT

\item In discrete time with \textit{unit} time steps, $K_{t+1} = I_t + (1-\delta) K_t$

\item With arbitrary $\Delta$ time step, $K_{t+\Delta} = K_t + \Delta (I_t - \delta K_t)$

\item Continuous-time limit:
\begin{align*}
	K_{t+\Delta} &= K_t + \Delta (I_t - \delta K_t) \\
	\frac{K_{t+\Delta} - K_t}{\Delta} &= I_t -\delta K_t \\
	\dot K_t &= I_t -\delta K_t
\end{align*}
\end{witemize}
\end{frame}



%%%%%%%%%%%%%%%%%%%%%%%%%%  SLIDE   %%%%%%%%%%%%%%%%%%%%%%%%%%%%%%%%
\begin{frame}{}

\begin{witemize}
\item Suppose $\{ I_t \}_{t \geq 0}$ exogenously given

\item Solving this \textit{inhomogeneous equation}, we use \textit{integrating factor}:
\begin{align*}
	\dot K_t + \delta K_t &= I_t  \\
	e^{\int_0^t \delta ds} \dot K_t + e^{\int_0^t \delta ds} \delta K_t &= e^{\int_0^t \delta ds} I_t
\end{align*}

\item Notice that $\int_0^t \delta ds = \delta \int_0^t ds = \delta [s]_0^t = \delta(t - 0) = \delta t$, so 
\begin{align*}
	e^{\delta t} \dot K_t + e^{\delta t} \delta K_t &= e^{\delta t} I_t
\end{align*}

\item We have $e^{\delta t} \dot K_t + e^{\delta t} \delta K_t = \frac{d}{dt} (K_t e^{\delta t})$, integrating:
\begin{align*}
	K_t e^{\delta t} &= \tilde C + \int_0^t e^{\delta s} I_s ds \\
	K_t &= C + \int_0^t e^{- \delta(t-s)} I_s ds
\end{align*}

\item Integrating constant solves initial condition: $C = K_0$
\end{witemize}
\end{frame}



%%%%%%%%%%%%%%%%%%%%%%%%%%  SLIDE   %%%%%%%%%%%%%%%%%%%%%%%%%%%%%%%%
\begin{frame}{}
\textbf{Wealth dynamics} (\textit{very important equation in this course}):
\begin{equation*}
\dot a_t = r_t a_t + y_t - c_t
\end{equation*}
\begin{witemize}
\item $r_t$ is the real rate of return on wealth, $y_t$ is income, and $c_t$ is consumption

\item Structure of the equation similar to capital accumulation equation
\end{witemize}
\end{frame}



%%%%%%%%%%%%%%%%%%%%%%%%%%  SLIDE   %%%%%%%%%%%%%%%%%%%%%%%%%%%%%%%%
\begin{frame}{}
\textbf{Consumption Euler equation}:
\begin{equation*}
\frac{\dot C_t}{C_t} = r_t - \rho
\end{equation*}
\begin{witemize}
\item The Euler equation typically takes the form of a \textit{backward equation} and comes with a terminal condition ($C_T$) or transversality condition ($\lim_{T\to\infty} C_T$)

\item Stationary point only if $r_t = \rho$

\item Suppose we are at $r_t = r = \rho$ and a shock is realized. $r_0 > r$ what happens? $r_0 < r$ what happens? 
\end{witemize}
\end{frame}


%%%%%%%%%%%%%%%%%%%%%%%%%%  SLIDE   %%%%%%%%%%%%%%%%%%%%%%%%%%%%%%%%
\begin{frame}{6. Example: Solow growth model}
\begin{witemize}
\item As before, $Y_t = C_t + I_t$ and 
\begin{equation*}
	\dot K_t = Y_t - C_t - \delta K_t
\end{equation*}

\item Representative firms operates neoclassical production function
\begin{equation*}
	Y_t = F(K_t, L_t, A_t)
\end{equation*}

\item Normalize labor to $L_t = 1$ and hold TFP constant $A_t = A$

\item We again assume constant savings rate: $Y_t - C_t = I_t = s Y_t$

\item Assume Cobb-Douglas $Y_t = A K_t^\alpha$ so equilibrium allocation
\begin{equation*}
	\dot K_t = s A K_t^\alpha - \delta K_t
\end{equation*}
\end{witemize}
\end{frame}


%%%%%%%%%%%%%%%%%%%%%%%%%%  SLIDE   %%%%%%%%%%%%%%%%%%%%%%%%%%%%%%%%
\begin{frame}{}
	
{\small
\begin{witemize}
\item Steady state is given by
\begin{equation*}
	K_{ss} = \bigg( \frac{sA}{\delta} \bigg)^\frac{1}{1-\alpha}
\end{equation*}

\item Key equilibrium condition in $\dot K_t$ is \textit{non-linear} --- how to proceed?

\item Let $X_t = K_t^{1-\alpha}$, then 
\begin{align*}
	\dot X_t &= (1-\alpha) K_t^{-\alpha} \dot K_t  \\
	&= (1-\alpha) K_t^{-\alpha} ( s A K_t^\alpha - \delta K_t )  \\
	&= (1-\alpha) s A - (1-\alpha) K_t^{1-\alpha} \delta \\
	&= (1-\alpha) s A - (1-\alpha) \delta X_t
\end{align*}

\item Solution with initial condition $X_0$ (work this out):
\begin{equation*}
	X_t = X_{ss} + e^{-(1-\alpha) \delta t} \bigg[ X_0 - X_{ss} \bigg], \;\;\; \text{ where } X_{ss} = \frac{sA}{\delta}
\end{equation*}

\item Transition dynamics (rate of convergence) governed by $-(1-\alpha) \delta$

\end{witemize}
}
\end{frame}



%%%%%%%%%%%%%%%%%%%%%%%%%%  SLIDE   %%%%%%%%%%%%%%%%%%%%%%%%%%%%%%%%
\begin{frame}{7. What are partial differential equations?}
\begin{witemize}
\item Partial differential equations (PDEs) generalize ODEs to higher-dimensional state spaces

\item PDEs are at the heart of (i) continuous-time \textbf{dynamic programming} and (ii) heterogeneous-agent models in macro

\item PDEs have long been a core tool in physics, applied math, ... \\
$\implies$ increasingly used in economics

\end{witemize}
\end{frame}


%%%%%%%%%%%%%%%%%%%%%%%%%%  SLIDE   %%%%%%%%%%%%%%%%%%%%%%%%%%%%%%%%
\begin{frame}{}
	
{\small
\begin{witemize}
\item Consider a function $u(x_1, x_2, \ldots, x_n)$ where $x_1, \ldots, x_n$ are coordinates in $\mathbb R^n$

\item Partial derivatives of $u(\cdot)$
\begin{equation*}
	\frac{\partial u}{\partial x_i} \equiv \partial_{x_i} u \hspace{5mm} \text{ and } \hspace{5mm} \frac{\partial^2 u}{\partial x_i \partial x_j} = \partial_{x_i x_j} u
\end{equation*}

\item A PDE is an equation in $u$ and its partial derivatives --- fully generally:
\begin{equation*}
	0 = G(u, \partial_{x_1} u, \ldots, \partial_{x_n} u, \partial_{x_1 x_1} u, \ldots )
\end{equation*}

\item The \textit{order} of the PDE, is the order of the highest partial derivative

\item Examples from physics
\begin{itemize}
	\item Heat equation: $\partial_t u = \partial_{xx} u$ (second-order, linear, homogeneous)
	
	\item Wave equation: $\partial_{tt} u = \partial_{xx} u$ (second-order, linear, homogeneous)
	
	\item Transport equation: $\partial_t u = \partial_x u$ (first-order, linear, homogeneous)
\end{itemize}

\item Income distribution ``solves heat equation'', wealth dynamics ``solve transport equations'', dynamic programming often transport + heat 
\end{witemize}
}
\end{frame}




%%%%%%%%%%%%%%%%%%%%%%%%%%  SLIDE   %%%%%%%%%%%%%%%%%%%%%%%%%%%%%%%%
\begin{transitionframe}
	{\color{white} \Large \textbf{Part 2: Optimization with Deterministic Dynamics} \vspace{2mm}}
\end{transitionframe}


%%%%%%%%%%%%%%%%%%%%%%%%%%  SLIDE   %%%%%%%%%%%%%%%%%%%%%%%%%%%%%%%%
\begin{frame}{1. Neoclassical growth model in continuous time}
\begin{witemize}
\item The lifetime value of the representative household is
\begin{equation*}
	v(k_0) = \max_{\{ c_t \}_{t \geq 0} } \int_0^\infty e^{-\rho t} u(c_t) dt
\end{equation*}
subject to
\begin{align*}
	\dot k_t &= F(k_t) - \delta k_t - c_t \\
	k_0 &\text{ given },
\end{align*}
where $\dot x_t = \frac{d}{dt} x_t$, $\rho$ is the discount rate, $c_t$ is the rate of consumption, $u(\cdot)$ is instantaneous utility flow, and $\dot k_t$ is the rate of (net) capital accumulation

\item No uncertainty for now

\item This is the \textbf{sequence problem} in continuous time
\end{witemize}
\end{frame}


%%%%%%%%%%%%%%%%%%%%%%%%%%  SLIDE   %%%%%%%%%%%%%%%%%%%%%%%%%%%%%%%%
\begin{frame}{2. Calculus of variations} 
\begin{witemize}
\item Resources:
\begin{itemize}
	\item LeVeque: Finite Difference Methods for Ordinary and Partial Differential Equations
	
	\item Kamien and Schwartz: Dynamic Optimization
	
	\item Gelfand and Fomin: Calculus of Variations
\end{itemize}

\item This dynamic optimization problem is associated with the Lagrangian
\begin{equation*}
	L = \int_0^\infty e^{-\rho t} \bigg[ u(c_t) + \mu_t \bigg( F(k_t) - \delta k_t - c_t - \dot k_t \bigg) \bigg] dt 
\end{equation*}

\item $\mu_t$ is the Lagrange multiplier on the capital accumulation ODE

\item What do we do with $\dot k_t$??

\end{witemize}
\end{frame}


%%%%%%%%%%%%%%%%%%%%%%%%%%  SLIDE   %%%%%%%%%%%%%%%%%%%%%%%%%%%%%%%%
\begin{frame}{}
\begin{witemize}		
\item Integrate by parts:
\begin{align*}
	\int_0^\infty e^{-\rho t} \mu_t \dot k_t dt &= e^{-\rho t} \mu_t k_t \Big|_0^\infty - \int_0^\infty \frac{d}{dt} \bigg( e^{-\rho t} \mu_t \bigg) k_t dt \\
	&= - \mu_0 k_0 + \int_0^\infty e^{-\rho t} \rho \mu_t k_t dt - \int_0^\infty e^{-\rho t} \dot \mu_t k_t dt
\end{align*}

\item Plugging into Lagrangian: 
\begin{equation*}
	L = \int_0^\infty e^{-\rho t} \bigg[ u(c_t) + \mu_t \bigg( F(k_t) - \delta k_t - c_t \bigg) - \rho \mu_t k_t + \dot \mu_t k_t \bigg] dt + \mu_0 k_0
\end{equation*}

\item What have we accomplished? 

\item Notice $\mu_0 k_0$, this is crucial. What's intuition? 

\end{witemize}
\end{frame}


%%%%%%%%%%%%%%%%%%%%%%%%%%  SLIDE   %%%%%%%%%%%%%%%%%%%%%%%%%%%%%%%%
\begin{frame}{}
\begin{equation*}
L = \int_0^\infty e^{-\rho t} \bigg[ u(c_t) + \mu_t \bigg( F(k_t) - \delta k_t - c_t \bigg) - \rho \mu_t k_t + \dot \mu_t k_t \bigg] dt + \mu_0 k_0
\end{equation*}

\begin{witemize}
\item The planner optimizes over paths $\{ c_t \}$ and $\{ k_t \}$

\item At an optimum, there cannot be \textit{any} small perturbation in these paths that the planner finds preferable

\item Let $\{ c_t \}$ and $\{ k_t \}$ be \textit{candidate} optimal paths. Consider $\hat c_t = c_t + \alpha h_t^c$ and $\hat k_t = k_t + \alpha h_t^k$ for arbitrary functions $h_t^c$ and $h_t^k$
\end{witemize}


\vspace{-2mm}
\begin{align*}
L(\alpha) = \int_0^\infty e^{-\rho t} \bigg[ &u(c_t + \alpha h_t^c) + \mu_t \bigg( F(k_t + \alpha h_t^k) - \delta k_t - \delta \alpha h_t^k - c_t - \alpha h_t^c \bigg) \\
&- \rho \mu_t (k_t + \alpha h_t^k) + \dot \mu_t (k_t + \alpha h_t^k) \bigg] dt + \mu_0 (k_0 + \alpha h_0^k)
\end{align*}

\vspace{-2mm}
\begin{witemize}
\item What about \textit{boundary conditions}? At $t=0$, capital stock is fixed ($k_0$ given) while consumption is free. So must have: $h_0^k = 0$ while $h_0^c$ is free
\end{witemize}
\end{frame}


%%%%%%%%%%%%%%%%%%%%%%%%%%  SLIDE   %%%%%%%%%%%%%%%%%%%%%%%%%%%%%%%%
\begin{frame}{}

Necessary condition for optimality: $\frac{d}{d \alpha} L(0) = 0$
\begin{align*}
L(\alpha) = \int_0^\infty e^{-\rho t} \bigg[ &u(c_t + \alpha h_t^c) + \mu_t \bigg( F(k_t + \alpha h_t^k) - \delta k_t - \delta \alpha h_t^k - c_t - \alpha h_t^c \bigg) \\
&- \rho \mu_t (k_t + \alpha h_t^k) + \dot \mu_t (k_t + \alpha h_t^k) \bigg] dt + \mu_0 (k_0 + \alpha h_0^k)
\end{align*}

Work this out yourselves (many times, in many applications!)
\begin{align*}
\frac{d}{d \alpha}L(0) = \int_0^\infty e^{-\rho t} \bigg[ &u'(c_t) h_t^c + \mu_t \bigg( F'(k_t) h_t^k  - \delta h_t^k - h_t^c \bigg) \\
&- \rho \mu_t h_t^k + \dot \mu_t h_t^k \bigg] dt + \mu_0 h_0^k
\end{align*}
where $h_0^k = 0$ because $k_0$ is fixed


\end{frame}



%%%%%%%%%%%%%%%%%%%%%%%%%%  SLIDE   %%%%%%%%%%%%%%%%%%%%%%%%%%%%%%%%
\begin{frame}{}

Group terms: 
\begin{align*}
\frac{d}{d \alpha}L(0) = \int_0^\infty e^{-\rho t} \bigg[ &\bigg( u'(c_t) - \mu_t \bigg) h_t^c + \bigg( \mu_t \Big( F'(k_t) - \delta \Big) - \rho \mu_t + \dot \mu_t \bigg) h_t^k \bigg] dt 
\end{align*}

\vspace{5mm}
\textbf{Fundamental Theorem of the Calculus of Variations}: Since $h_t^c$ and $h_t^k$ were arbitrary, we must have \textit{pointwise}
\begin{align*}
0 &= u'(c_t) - \mu_t \\
0 &= \mu_t \Big( F'(k_t) - \delta \Big) - \rho \mu_t + \dot \mu_t
\end{align*}

\vspace{5mm}
\textbf{Proposition.} (Euler equation for marginal utility) 
\begin{equation*}
\frac{\dot \mu_t}{\mu_t} = \frac{\dot u_{c, t}}{u_{c, t}} = \rho -  F'(k_t) + \delta = \rho - r_t 
\end{equation*}

\end{frame}


%%%%%%%%%%%%%%%%%%%%%%%%%%  SLIDE   %%%%%%%%%%%%%%%%%%%%%%%%%%%%%%%%
\begin{frame}{}

\begin{witemize}
\item We have now solved the neoclassical growth model in continuous time. Its solution is given by a system of two ODEs. 

\item Suppose $u(c) = \log(c)$ and $F(k) = k^\alpha$, then:
\begin{align*}
	\frac{\dot c_t}{c_t} &= \alpha k_t^{\alpha - 1} - \delta - \rho \\
	\dot k_t &= k_t^\alpha - \delta k_t - c_t
\end{align*}
with $k_0$ given

\item Derive the consumption Euler equation yourselves!

\item What are the boundary conditions? (Always ask about BCs!) 
\begin{itemize}
	\item Initial condition on capital: $k_0$ given
	\item Terminal condition on consumption : $\lim_{T \to \infty} c_T = c_\text{ss}$ 
\end{itemize}

\end{witemize}

\end{frame}



%%%%%%%%%%%%%%%%%%%%%%%%%%  SLIDE   %%%%%%%%%%%%%%%%%%%%%%%%%%%%%%%%
\begin{frame}{3. Optimal control theory}
\begin{witemize}
\item Optimal control theory emerged from the calculus of variations

\item Applies to dynamic optimization problems in continuous time that feature (ordinary) differential equations as constraints

\item Again the neoclassical growth model:
\begin{equation*}
	v(k_0) = \max_{\{ c_t \}_{t \geq 0} } \int_0^\infty e^{-\rho t} u(c_t) dt
\end{equation*}
subject to
\begin{align*}
	\dot k_t &= F(k_t) - \delta k_t - c_t,  \quad k_0 \text{ given}
\end{align*}

\item Three new terms: 
\begin{itemize}
	\item \textbf{State variable}: $k_t$
	
	\item \textbf{Control variable}: $c_t$
	
	\item \textbf{Hamiltonian}: $H(c_t, k_t, \mu_t) = u(c_t) + \mu_t \big[ F(k_t) - \delta k_t - c_t \big]$
\end{itemize}
\end{witemize}	
\end{frame}



%%%%%%%%%%%%%%%%%%%%%%%%%%  SLIDE   %%%%%%%%%%%%%%%%%%%%%%%%%%%%%%%%
\begin{frame}{}
\begin{witemize}
\item With Hamiltonian in hand, \textit{copy-paste} formula that we can always use:

\begin{itemize}
\item \textbf{Optimality condition}: $\frac{\partial}{\partial c} H = 0$
\item \textbf{Multiplier condition}: $\rho \mu_t - \dot \mu_t = \frac{\partial}{\partial k} H$
\item \textbf{State condition}: $\dot k_t = \frac{\partial}{\partial \mu} H$
\end{itemize}

\item This gives us the same equations that we derived using calc of variations:
\begin{align*}
	u'(c_t) &= \mu_t \\
	\rho \mu_t - \dot \mu_t &= \mu_t (F'(k_t) - \delta) \\
	\dot k_t &= F(k_t) - \delta k_t - c_t
\end{align*}

\item We again get system of Euler equation and capital accumulation:
\begin{align*}
	\dot c_t &= \frac{u'(c_t)}{u''(c_t)} \Big( \rho - F'(k_t) + \delta \Big) \\
	\dot k_t &= F(k_t) - \delta k_t - c_t
\end{align*}

\end{witemize}
\end{frame}




%%%%%%%%%%%%%%%%%%%%%%%%%%  SLIDE   %%%%%%%%%%%%%%%%%%%%%%%%%%%%%%%%
\begin{frame}{4. Simple example [\textit{skip}]}
\begin{witemize}

\item Credit: Kamien-Schwartz p. 129

\item Simple problem: not much intuition, but illustrates mechanics
\begin{equation*}
	\max \int_0^1 (x + u) dt 
\end{equation*}
subject to $\dot x = 1 - u^2$ and initial condition $x_0 = 1$

\item Step 1: form Hamiltonian $H(t, x, u, \lambda) = x + u + \lambda (1 - u^2)$

\item Step 2: necessary conditions (note: no discounting here)
\begin{align*}
	0 &= H_u = 1 - 2 \lambda u \\
	- \dot \lambda &= H_x = 1
\end{align*}
and terminal condition $\lambda_1 = 0$ (because $u_1$ is \textit{free})

\end{witemize}
\end{frame}



%%%%%%%%%%%%%%%%%%%%%%%%%%  SLIDE   %%%%%%%%%%%%%%%%%%%%%%%%%%%%%%%%
\begin{frame}{}
\begin{witemize}

\item Step 3: manipulate necessary conditions:
\begin{align*}
	\lambda &= 1 - t \\
	u &= \frac{1}{2 \lambda}
\end{align*}
and therefore: $u = \frac{1}{2} (1 - t)$

\item Finally: solve for all paths (control, state, multiplier)
\begin{align*}
	x_t &= t - \frac{1}{4} (1 - t) + \frac{5}{4} \\
	\lambda_t &= 1 - t \\
	u_t &= \frac{1}{2} (1 - t)
\end{align*}

\end{witemize}
\end{frame}



%%%%%%%%%%%%%%%%%%%%%%%%%%  SLIDE   %%%%%%%%%%%%%%%%%%%%%%%%%%%%%%%%
\begin{frame}{5. Hamilton-Jacobi-Bellman equation}
\begin{witemize}
\item Recall the neoclassical growth model in continuous time
\begin{equation*}
	v(k_0) = \max_{\{ c_t \}_{t \geq 0} } \int_0^\infty e^{-\rho t} u(c_t) dt
\end{equation*}
subject to
\begin{align*}
	\dot k_t &= F(k_t) - \delta k_t - c_t \\
	k_0 &\text{ given },
\end{align*}
where $\dot x_t = \frac{d}{dt} x_t$, $\rho$ is the discount rate, $c_t$ is the rate of consumption, $u(\cdot)$ is instantaneous utility flow, and $\dot k_t$ is the rate of (net) capital accumulation

\item No uncertainty for now %: $\{ r_t, y_t \}_{t \geq 0}$ are deterministic processes

\item This is the infinite-horizon sequence problem, $t \in [0, \infty)$

\item A function $v(\cdot)$ that solves this problem is a solution to the neoclassical growth model 
\end{witemize}
\end{frame}


%%%%%%%%%%%%%%%%%%%%%%%%%%  SLIDE   %%%%%%%%%%%%%%%%%%%%%%%%%%%%%%%%
\begin{frame}{}
\begin{witemize}
\item We will now work towards a recursive representation (good reference: Stokey textbook)

\item The discrete-time Bellman equation would be
\begin{equation*}
	v(k_t) = \max_c \Big\{ u(c) \Delta + \frac{1}{1 + \rho \Delta} v(k_{t+\Delta}) \Big\}
\end{equation*}
where $\beta = \frac{1}{1+ \rho \Delta}$

\item Next: multiply by $1 + \rho \Delta$ 
\begin{align*}
	(1 + \rho \Delta) v(k_t) &= \max_c \Big\{ (1 + \rho \Delta) u(c) \Delta + v(k_{t+\Delta}) \Big\} \\
	\rho \Delta v(k_t) &= \max_c \Big\{ u(c) \Delta + \rho u(c) \Delta^2 + v(k_{t+\Delta}) - v(k_t) \Big\} \\
	\rho v(k_t) &= \max_c \Big\{ u(c) + \rho u(c) \Delta + \frac{v(k_{t+\Delta}) - v(k_t)}{\Delta} \Big\}
\end{align*}
\end{witemize}
\end{frame}


%%%%%%%%%%%%%%%%%%%%%%%%%%  SLIDE   %%%%%%%%%%%%%%%%%%%%%%%%%%%%%%%%
\begin{frame}{}
\begin{align*}
	\rho v(k_t) &= \max_c \Big\{ u(c) + \rho u(c) \Delta + \frac{v(k_{t+\Delta}) - v(k_t)}{\Delta} \Big\}
\end{align*}

\vspace{4mm}
\begin{witemize}
\item Now we take limit $\Delta \to 0$ 

\item Notice $\rho u(c) \Delta \to 0$ and also
\begin{equation*}
	\lim_{\Delta \to 0} \frac{v(k_{t+\Delta}) - v(k_t)}{\Delta} =
	\lim_{\Delta \to 0} \frac{v(k(t+\Delta)) - v(k(t))}{\Delta} =
	\frac{d v(k(t))}{dt}
\end{equation*}

\item Therefore we arrive at
\begin{equation*}
	\rho v(k(t)) = \max_c \Big\{ u(c) + \frac{dv(k(t))}{dt} \Big\}
\end{equation*}
\end{witemize}
\end{frame}



%%%%%%%%%%%%%%%%%%%%%%%%%%  SLIDE   %%%%%%%%%%%%%%%%%%%%%%%%%%%%%%%%
\begin{frame}{}
\begin{witemize}
\item Only step left for us to do: What is $\frac{dv(k(t))}{dt}$?

\item Simply use Chain rule! $\frac{dv(k(t))}{dt} = \frac{dv}{dk} \frac{dk}{dt}$ and recall $\frac{dk}{dt} = F(k) - \delta k - c$

\item Therefore, we arrive at the \textbf{Hamilton-Jacobi-Bellman equation}:
\begin{equation*}
	\rho v(k) = \max_c \Big\{ u(c) + \Big( F(k) - \delta k - c \Big) v'(k) \Big\}
\end{equation*}

\item We drop $t$ subscripts for clarity: this equation holds for all possible $k$

\item Notice: We conjectured a stationary value function (what does this mean?)
\end{witemize}
\end{frame}



%%%%%%%%%%%%%%%%%%%%%%%%%%  SLIDE   %%%%%%%%%%%%%%%%%%%%%%%%%%%%%%%%
\begin{frame}{6. First-order condition for consumption}
\begin{itemize}
\item HJB still has ``max'' operator:
\begin{equation*}
	\rho v(k) = \max_c \Big\{ u(c) + \Big( F(k) - \delta k - c \Big) v'(k) \Big\}
\end{equation*}

\item To get rid of this, we have to resolve optimal consumption choice

\item First-order condition:
\begin{equation*}
	u'(c(k)) = v'(k)
\end{equation*}

\item This defines the \textbf{consumption policy function}

\item We can now plug back in, obtaining an ODE in $v'(k)$
\begin{equation*}
	\rho v(k) = u(c(k)) + \Big( F(k) - \delta k - c(k) \Big) v'(k)
\end{equation*}

\item Why is this a “stationary” value function and ODE? What would a time-dependent ODE look like? When would we get one?
\end{itemize}
\end{frame}



%%%%%%%%%%%%%%%%%%%%%%%%%%  SLIDE   %%%%%%%%%%%%%%%%%%%%%%%%%%%%%%%%
\begin{frame}{7. Envelope condition and Euler equation}

{\small
\begin{itemize}
\item We now derive the Euler equation in continuous time

\item We start with the \textbf{HJB envelope condition}. Differentiating in $k$:
\begin{align*}
	\rho v'(k) &= u'(c(k)) c'(k) + \Big( F'(k) - \delta - c'(k) \Big) v'(k) + \Big( F(k) - \delta k - c(k) \Big) v''(k) \\
	\rho v'(k) &= \Big( \underbrace{ F'(k) - \delta}_\text{interest rate $r$} \Big) v'(k) + \Big( F(k) - \delta k - c(k) \Big) v''(k) \\
	(\rho - r) v'(k) &= \Big( F(k) - \delta k - c(k) \Big) v''(k)
\end{align*}

\item Next, we characterize \textit{process} $d v'(k)$. Using Ito's lemma (even though no uncertainty):
\begin{align*}
	d v'(k) &= v''(k) dk \\
	&= v''(k) (F(k) - \delta k - c(k)) dt \\
	&= (\rho - r) v'(k) dt.
\end{align*}
\end{itemize}
}
\end{frame}


%%%%%%%%%%%%%%%%%%%%%%%%%%  SLIDE   %%%%%%%%%%%%%%%%%%%%%%%%%%%%%%%%
\begin{frame}{}

{\small
\begin{itemize}
\item Recall first-order condition $u'(c(k)) = v'(k)$.

\item The \textbf{Euler equation for marginal utility }is given by
\begin{equation*}
	\frac{d u'(c)}{u'(c)} = (\rho - r) dt.
\end{equation*}

\item To go from marginal utility to consumption, we use CRRA utility: $u(c) = \frac{1}{1-\gamma} c^{1-\gamma}$. $u'(c) = c^{-\gamma}$ is a function of \textit{process} $c$, so by Ito's lemma: 
\begin{align*}
	d u'(c) &= -\gamma c^{-\gamma - 1} dc \\
	&= -\gamma u'(c) \frac{dc}{c} 
\end{align*}

\item Plugging in yields \textbf{Euler equation for consumption} in continuous time:
\begin{equation*}
	\frac{dc}{c} = \frac{r - \rho}{\gamma} dt
\end{equation*}
or (you'll often see this notation when no uncertainty): $\frac{\dot c}{c} = \frac{r - \rho}{\gamma}$
\end{itemize}
}
\end{frame}



%%%%%%%%%%%%%%%%%%%%%%%%%%  SLIDE   %%%%%%%%%%%%%%%%%%%%%%%%%%%%%%%%
\begin{frame}{}

Connection between calculus of variations and HJB:
\begin{witemize}
\item What is the connection between costate / multiplier $\mu_t$ and marginal value of wealth $V'(k)$?

\item What is the connection between multiplier equation and envelope condition?
\end{witemize}
\end{frame}


%%%%%%%%%%%%%%%%%%%%%%%%%%  SLIDE   %%%%%%%%%%%%%%%%%%%%%%%%%%%%%%%%
\begin{frame}{9. Boundary conditions}

{\small
\begin{itemize}
\item Important: everything we have done so far is only valid in the \textbf{interior of the state space}

\item What's the state space of a model? 

\item For the neoclassical growth model without uncertainty, state space is $k \in [0, \infty)$, or 
\begin{equation*}
	\Big\{ k \mid k \in [0, \bar k] \Big\}
\end{equation*}
where we impose an upper boundary $\bar k$ (think about putting this on the computer)

\item This is like the domain of the function $v(k)$ that will be valid 

\item We say $\{ 0, \bar k \}$ is the \textbf{boundary} of the state space and $(0, \bar k)$ is the \textbf{interior}

\item As is the case \textbf{for all differential equations}, the HJB holds on the interior and we need \textbf{boundary conditions} to characterize $v(k)$ along the boundary 
\end{itemize}
}
\end{frame}



%%%%%%%%%%%%%%%%%%%%%%%%%%  SLIDE   %%%%%%%%%%%%%%%%%%%%%%%%%%%%%%%%
\begin{frame}{}

{\small
\begin{witemize}
\item What differential equation is HJB in this model? How many boundary conditions do we need?

\item In terms of the economics, what is correct economic behavior at the boundary $k \in \{0, \bar k\}$? 

\pause
\item We want households to not leave the state space, so we impose that they do not dissave / borrow as $k \to 0$ and save as $k \to \bar k$

\item Nice intuition: 2 boundary inequalities do same job as 1 boundary equality

\item This implies: (why?)
\begin{align*}
	u'(c(0)) &\geq v'(0) \\
	u'(c(\bar k)) &\leq v'(\bar k)
\end{align*}

\item In neoclassical growth model, Inada conditions take care of this (you never reach boundary) 

\end{witemize}
}
\end{frame}




%%%%%%%%%%%%%%%%%%%%%%%%%%  SLIDE   %%%%%%%%%%%%%%%%%%%%%%%%%%%%%%%%
\begin{transitionframe}
	{\color{white} \Huge \textbf{Part 3: Applications} \vspace{2mm}}
\end{transitionframe}


%%%%%%%%%%%%%%%%%%%%%%%%%%  SLIDE   %%%%%%%%%%%%%%%%%%%%%%%%%%%%%%%%
\begin{frame}{1. Labor: search and matching}

\begin{witemize}
\item One of most important ideas in labor: frictional search and matching 

	{\footnotesize Diamond-Mortensen-Pissarides (DMP) model}

\item This is just a simple application of dynamic programming
\end{witemize}

\end{frame}


%%%%%%%%%%%%%%%%%%%%%%%%%%  SLIDE   %%%%%%%%%%%%%%%%%%%%%%%%%%%%%%%%
\begin{frame}{}

\textbf{Firms:} Can post vacancies at cost (rate) $c$, vacancy filled at rate $q$. Value of a vacancy is given by HJB
\begin{equation*}
	r V = - c + q(J-V).
\end{equation*}
Assume firms post until $V = 0$. Once matched, workers produce revenue at rate $p$ and cost wage $w$. Match separates at rate $s$. Value of job is given by HJB 
\begin{equation*}
	r J = p - w - s J
\end{equation*}

\vspace{5mm}
\textbf{Labor demand:} 
In equilibrium, labor demand schedule given by
\begin{equation*}
	p-w = (r+s) \frac{c}{q}
\end{equation*}
Profit $p-w$ equalized with amortized cost of search / posting vacancies 

\end{frame}

%%%%%%%%%%%%%%%%%%%%%%%%%%  SLIDE   %%%%%%%%%%%%%%%%%%%%%%%%%%%%%%%%
\begin{frame}{}

\textbf{Workers:}
When worker is unemployed, gets benefit $b$ and can search at intensity $\lambda$, which costs $\psi(\lambda)$. When employed, gets wage $w$ but separates at rate $s$. Let $U$ be value of unemployment and $E$ value of employment:
\begin{align*}
	r U &= \max_\lambda \Big\{ b - \psi(\lambda) + \lambda(E - U) \Big\} \\
	r E &= w + s(U - E)
\end{align*}

\vspace{5mm}
\textbf{Labor supply schedule} characterized by FOC for search intensity
\begin{align*}
	\psi'(\lambda) = E - U
\end{align*}
where $E$ and $U$ solve the coupled system of HJBs

\end{frame}


%%%%%%%%%%%%%%%%%%%%%%%%%%  SLIDE   %%%%%%%%%%%%%%%%%%%%%%%%%%%%%%%%
\begin{frame}{2. Urban / trade / dynamic spatial: migration}
\begin{witemize}
\item One of most important themes in urban, trade, international and dynamic spatial literatures: people move (migrate) in response to shocks

	{\footnotesize For example: To what extent do households migrate in response to China trade shock or climate change?}

\item Turns out: state-of-the-art dynamic migration model (Caliendo-Dvorkin-Parro) is a simple application of our tools

\end{witemize}
\end{frame}


%%%%%%%%%%%%%%%%%%%%%%%%%%  SLIDE   %%%%%%%%%%%%%%%%%%%%%%%%%%%%%%%%
\begin{frame}{}

\textbf{Households:} 
There are $N$ regions indexed by $j$. Consider a household $i$ and denote her region $j_{i, t}$. Lifetime utility is 
\begin{equation*}
	V_{i, 0} = \max \mathbb E \int_0^\infty e^{- \rho t} u(c_{i, t}) dt
\end{equation*}

\begin{witemize}
\item Household inelastically supplies $1$ unit of labor, earns wage $w_{j_{i, t}}$

\item For simplicity: They cannot save or borrow, so $c_{i, t} = w_{j_{i, t}}$ (``hand-to-mouth'')

\item Problem will be stationary because $w_j$ are time-invariant
\end{witemize}
\end{frame}


%%%%%%%%%%%%%%%%%%%%%%%%%%  SLIDE   %%%%%%%%%%%%%%%%%%%%%%%%%%%%%%%%
\begin{frame}{}

\textbf{Migration:} discrete-choice optimal stopping problem
\vspace{4mm}
\begin{witemize}
\item Households face fixed cost $\kappa_{jk}$ to move from $j$ to $k$

\item \textbf{Key trick:} At rate $\mu$, household draws opportunity and \textbf{extreme-value taste shock} $\epsilon_k$ with shape parameter $\theta$ for possible destinations $k$
\end{witemize}

\vspace{5mm}
\textbf{Recursive representation:}
\begin{equation*}
	\rho V_j = u(w_j) 
	+ \mu \Big( \mathbb E \Big[ \max_k V_k - \kappa_{jk} + \epsilon_k \Big] - V_j \Big) 
\end{equation*}

\begin{witemize}
\item Using extreme-value taste shocks $\to$ nice expression for migration flows

\item Incredibly easy to solve on computer using tools you're learning in Section
\end{witemize}
\end{frame}


%%%%%%%%%%%%%%%%%%%%%%%%%%  SLIDE   %%%%%%%%%%%%%%%%%%%%%%%%%%%%%%%%
\begin{frame}{3. Macro: sticky prices} 
\begin{witemize}
\item In the data, firms do not adjust prices instantly 

\item Price stickiness (nominal rigidities) is at the heart of the New Keynesian model 

\item We will derive the New Keynesian Phillips Curve as simple application of our tools $\implies$ much easier to derive in continuous time!!

\item Consider a continuum of firms indexed by $j$ that compete monopolistically

\item Firm $j$ faces demand function 
\begin{equation*}
	Y_{j, t} = \bigg( \frac{P_{j, t}}{P_t} \bigg)^{- \epsilon} Y_t
\end{equation*}
where $Y_t$ and $P_t$ are aggregate (industry) demand and price index
\end{witemize}
\end{frame}


%%%%%%%%%%%%%%%%%%%%%%%%%%  SLIDE   %%%%%%%%%%%%%%%%%%%%%%%%%%%%%%%%
\begin{frame}{}
\begin{witemize}
\item Firms produce intermediate varieties with the linear production function 
\begin{equation*}
	Y_{j, t} = A_t N_{j, t}
\end{equation*}

\item $A_t$ is aggregate productivity and $N_{j, t}$ firm $j$'s labor demand 

\item Firm $j$ sells at price $P_{j, t}$, profit = revenue net of operating expenses
\begin{equation*}
	\Pi_{j, t} = P_{j, t} Y_{j, t} - W_t N_{j, t}
\end{equation*}

\item Firms maximize NPV of future profit streams, discounted at interest rate $r$
\end{witemize}
\end{frame}


%%%%%%%%%%%%%%%%%%%%%%%%%%  SLIDE   %%%%%%%%%%%%%%%%%%%%%%%%%%%%%%%%
\begin{frame}{}
\begin{witemize}
\item Firms set prices optimally over time by choosing inflation $\dot P_{j, t} = P_{j, t} \pi_{j, t}$
\item Firms pay quadratic adjustment cost $\frac{\delta}{2} \pi_{j, t}^2 P_t Y_t$ to adjust nominal price

\item Firm problem:
\begin{equation*}
	\max_{ \{ \pi_{j, t}, N_{j, t} \}_{t \geq 0} } \int_0^\infty e^{- r ds} \bigg( P_{j, t} Y_{j, t} - W_t N_{j, t} - \frac{\delta}{2} \pi_{j, t}^2 P_t \bigg) dt,
\end{equation*}

\item Firms are small and take as given $\{ W_t, Y_t, P_t \}_{t \geq 0}$ and initial condition $P_{j, 0}$

\item Any two firms $j$ and $j'$ with same initial price $P_{j, 0} = P_{j', 0}$ adopt identical inflation and production policies $\implies$ we get back to representative firm

\end{witemize}
\end{frame}


%%%%%%%%%%%%%%%%%%%%%%%%%%  SLIDE   %%%%%%%%%%%%%%%%%%%%%%%%%%%%%%%%
\begin{frame}{}
\begin{witemize}
\item Hamiltonian (state: $P_{j, t}$, control: $\pi_{j, t}$, multiplier: $\eta_{j, t}$):
\begin{equation*}
	\mathcal H_t(P_{j, t}, \pi_{j, t}, \eta_{j, t}) = P_{j, t}^{1-\epsilon} P_t^{\epsilon} Y_t - \frac{W_t}{A_t} P_{j, t}^{-\epsilon} P_t^{\epsilon} Y_t  - \frac{\delta}{2} \pi_{j, t}^2 P_t Y_t + \eta_{j, t} P_{j, t} \pi_{j, t}
\end{equation*}

\item Conditions for optimum:
\begin{align*}
	i_t \eta_{j, t} - \dot \eta_{j, t} &= (1-\epsilon) P_{j, t}^{-\epsilon} P_t^{\epsilon} Y_t + \epsilon \frac{W_t}{A_t} P_{j, t}^{-\epsilon - 1} P_t^{\epsilon} Y_t + \eta_{j, t} \pi_{j, t}  \\
	0 &= - \delta \pi_{j, t} P_t Y_t + \eta_{j, t} P_{j, t},
\end{align*}
as well as the initial condition for the multiplier $\eta_{j, 0} = 0$

\end{witemize}
\end{frame}


%%%%%%%%%%%%%%%%%%%%%%%%%%  SLIDE   %%%%%%%%%%%%%%%%%%%%%%%%%%%%%%%%
\begin{frame}{}
\begin{witemize}

\item Now we can impose symmetric equilibrium: $P_{j, t} = P_t$ for all $j$
\begin{align*}
	i_t \eta_t - \dot \eta_t  &= (1-\epsilon) P_t^{-\epsilon} P_t^{\epsilon} Y_t + \epsilon \frac{W_t}{A_t} P_t^{-\epsilon - 1} P_t^{\epsilon} Y_t + \eta_t \pi_t  \\
	0 &= - \delta \pi_t P_t Y_t + \eta_t P_t
\end{align*}

\item Or simply:
\begin{align*}
	i_t \eta_t - \dot \eta_t  &= (1-\epsilon) Y_t + \epsilon \frac{w_t}{A_t} Y_t + \eta_t \pi_t  \\
	\eta_t &= \delta \pi_t Y_t
\end{align*}

\item Differentiating eq. 2 ($\dot \eta_t = \delta \dot \pi_t Y_t + \delta \pi_t \dot Y_t$) yields:
\begin{equation*}
	 \dot \pi_t = \pi_t \bigg( i_t -  \pi_t - \frac{\dot Y_t}{Y_t} \bigg) - \frac{\epsilon}{\delta} \bigg( \frac{w_t}{A_t} - \frac{\epsilon - 1}{\epsilon} \bigg)
\end{equation*}
\end{witemize}
\end{frame}


%%%%%%%%%%%%%%%%%%%%%%%%%%  SLIDE   %%%%%%%%%%%%%%%%%%%%%%%%%%%%%%%%
\begin{frame}{}
\begin{witemize}

\item From previous slide: 
\begin{equation*}
	 \dot \pi_t = \pi_t \bigg( i_t -  \pi_t - \frac{\dot Y_t}{Y_t} \bigg) - \frac{\epsilon}{\delta} \bigg( \frac{w_t}{A_t} - \frac{\epsilon - 1}{\epsilon} \bigg)
\end{equation*}

\item Last step: recall Euler equation of the representative household
\begin{equation*}
	\frac{\dot C_t}{C_t} = r_t - \rho
\end{equation*}
and use goods market clearing
\begin{equation*}
	Y_t = C_t
\end{equation*}

\item \textbf{NKPC:}
\begin{equation*}
	 \dot \pi_t = \rho \pi_t - \frac{\epsilon}{\delta} \bigg( \frac{w_t}{A_t} - \frac{\epsilon - 1}{\epsilon} \bigg)
\end{equation*}
\end{witemize}
\end{frame}


%%%%%%%%%%%%%%%%%%%%%%%%%%  SLIDE   %%%%%%%%%%%%%%%%%%%%%%%%%%%%%%%%
\begin{frame}{4. IO: duopoly} 

\begin{witemize}
\item Consider continuous-time variant of Ericson-Pakes (1995) quality-ladder model 

\item Duopolistic competition: $2$ firms $i \in \{A, B\}$ produce good with quality $q_t^i$ and maximize NPV of profits: $\max \int_0^\infty e^{- rt} \pi_t^i dt$. They compete over investments $\iota_t^i$:
\begin{equation*}
	\dot q^i = \iota_t^i - \delta q_t^i
\end{equation*}


\item Profits $\pi_t^i$ depend on both firms' product qualities $\rightarrow$ state variables for recursive representation are $\omega \equiv (\omega^A, \omega^B)$

\item Best-response of firm $A$ to firm $B$ characterized by HJB
\begin{equation*}
	r V^A(\omega) = \pi^A(\omega) + \max_\iota \Big\{ (\iota - \delta \omega^A) V_{\omega^A}^A(\omega) - \Phi(\iota) \Big\} + (\iota^B - \delta \omega^B) V_{\omega^B}^A(\omega) 
\end{equation*}
where $\Phi(\cdot)$ is cost of investment, and best-response takes $\iota^B$ as given
\end{witemize}
\end{frame}


%%%%%%%%%%%%%%%%%%%%%%%%%%  SLIDE   %%%%%%%%%%%%%%%%%%%%%%%%%%%%%%%%
\begin{frame}{5. Public finance: tax competition}

{\small
\begin{witemize}
\item Two countries, $i \in \{A, B\}$, setting corporate tax rates $\tau_t^i$ on firms operating / headquartered in country $i$

\item Mass of multinational firms $j$, with $\mu_t$ denoting $\%$ in country $A$ at time $t$

\item Firms relocate activity / headquarters at rate $\theta$ towards low-tax country:
\begin{equation*}
	d \mu_t = \theta \mu_t (\tau_t^B - \tau_t^A)^\gamma dt  
\end{equation*}

\item Country $A$ maximizes tax revenue: $\max \int_0^\infty e^{- \rho t} \tau_t^A \mu_t dt$. Countries compete over taxes $\{ \tau_{it} \}$

\item Dynamic Nash: country $A$ sets $\tau_t^A$ as best response taking $\tau_t^B$ as given 

\item Recursive representation: the only state variable is $\mu_t$
\begin{equation*}
	\rho V^A(\mu) = \max_{\tau^A} \Big\{ \tau^A \mu + \theta \mu \Big( \tau^B(\mu) - \tau^A \Big)^\gamma \partial_\mu V^A(\mu) \Big\}
\end{equation*}
Best response strategies: $0 = \mu + \gamma \theta \mu (\tau^B(\mu) - \tau^A)^{\gamma - 1} V_\mu^A(\mu)$
\end{witemize}
}
\end{frame}



\end{document}


%%%%%%%%%%%%%%%%%%%%%%%%%%  SLIDE   %%%%%%%%%%%%%%%%%%%%%%%%%%%%%%%%
\begin{frame}{2. Urban / trade / dynamic spatial: migration}
\begin{witemize}
\item One of most important themes in urban, trade, international and dynamic spatial literatures: people move (migrate) in response to shocks

	{\footnotesize For example: To what extent do households migrate in response to China trade shock or climate change?}

\item Turns out: state-of-the-art dynamic migration model (Caliendo-Dvorkin-Parro) is a simple application of our tools

\item Here: also add dynamic consumption-savings problem
\end{witemize}
\end{frame}


%%%%%%%%%%%%%%%%%%%%%%%%%%  SLIDE   %%%%%%%%%%%%%%%%%%%%%%%%%%%%%%%%
\begin{frame}{}

\textbf{Households:} 
There are $N$ regions indexed by $j$. Consider a household $i$ and denote her region $j_{i, t}$. Lifetime utility is 
\begin{equation*}
	V_{i, 0} = \max \mathbb E \int_0^\infty e^{- \rho t} u(c_{i, t}) dt
\end{equation*}

\begin{witemize}
\item Household inelastically supplies $1$ unit of labor, earns wage $w_{j_{i, t}}$

\item They have a checking account and face budget constraint 
\begin{equation*}
	\dot a_{i, t} = r a_{i, t} + w_{j_{i, t}} - c_{i, t}
\end{equation*}

\item Problem will be stationary because $r$ and $w_j$ are time-invariant
\end{witemize}
\end{frame}


%%%%%%%%%%%%%%%%%%%%%%%%%%  SLIDE   %%%%%%%%%%%%%%%%%%%%%%%%%%%%%%%%
\begin{frame}{}

\textbf{Migration:} discrete-choice optimal stopping problem
\vspace{4mm}
\begin{witemize}
\item Households face fixed cost $\kappa_{jk}$ to move from $j$ to $k$

\item \textbf{Key trick:} At rate $\mu$, household draws opportunity and \textbf{extreme-value taste shock} $\epsilon_k$ with shape parameter $\theta$ for possible destinations $k$
\end{witemize}

\vspace{5mm}
\textbf{Recursive representation:}
\begin{equation*}
	\rho V(j, a) = \max_c \Big\{ u(c) 
	\underbrace{\vphantom{\max_k}
		+ (r a + w_j - c) V_a(j, a) 
	}_\text{consumption-savings}
	\underbrace{
		+ \mu \Big( \mathbb E \Big[ \max_k V(k, a) - \kappa_{jk} + \epsilon_k \Big] - V(j, a) \Big) \Big\}
	}_\text{migration}
\end{equation*}
\end{frame}

